\documentclass[12pt]{article}

\usepackage{fullpage,url,xspace,enumerate,multirow,enumitem}
\usepackage[pdftitle={CLAS 205 Notes},%
pdfsubject={University of Waterloo, CLAS 205, Winter 2017},%
pdfauthor={Daniel Cardoza}]{hyperref}

\begin{document}

\begin{center}
	{\Large\bf CLAS 205 : Medieval Society}\\
	\vspace{2mm}
	{\Large\bf Winter 2017}\\
	\vspace{4mm}
	{Daniel Cardoza}\\
\end{center}

\section*{Reading 1 : The Grandeur of Rome}

\subsection*{Overview}

\subsection*{Notes}

\subsection*{Question}

\section*{Reading 7 : The Raids of the Northmen}

\subsection*{Overview}
\begin{itemize}
	\item In the 9th, 10th centuries, Arabs from the South, Northmen from Scandinavia and Magyars from the Eastern Steppe plagued the towns of Europe (England, France, Low Countries)
\end{itemize}

\subsection*{Notes}

\subsubsection*{The Early Raids of the Northmen, 834 -859}
\begin{itemize}
	\item Northmen pillaged Nantes in 834, killing the bishop and clergy
	\item Northmen pillaged along the Garonne until Toulouse
	\item With 100 ships, the Northmen assailed Paris after being unopposed. Charles the Bald gave them 7000lbs of (coin, resources) for them to leave. They did, but pillaged all the way to the coast
	\item Danish pirates pillaged Nantes in 853-854 pillaged the city of Nantes, chateau of Blois before being routed at Orleans. Routed by the bishop of Orleans and Chartres
	\item Tried to pillage Poitiers in 855 but were defeated by Aquitanians
	\item Pillaged Orleans in 856
	\item Pillaged along the Rhone in 859, settling on the island of Camargue. Ravaged up until Valence. Travelled to Italy afterward
\end{itemize}

\subsubsection*{The Siege of Paris, 885}
\begin{itemize}
	\item Sigfred, leader of the Danes, assailed Paris with 700 ships in 885
	\item Count Odo led the defence of Paris
	\item After the first day of fighting, the town's tower was damaged and repaired
	\item There was a plague inside the city. Count Odo went to seek Charles, the Frankish Emperor for help
	\item Count Odo came back with Emperor Charles the Fat. Charles allowed theNorthmean to have Sens and gave them 700lbs of silver.
\end{itemize}
	
\subsection*{Question}

\begin{center}
	\textit{Did the cities meet the Viking raids with any type of resistance? Who led the resistance in Paris? What role did ordinary townspeople play in the defense of Paris? How did Emperor Charles the Fat choose to deal with the attackers?}
\end{center}

\begin{itemize}
	\item Certain cities did, but the reading seems to suggest they pillaged most towns unopposed. 
	\item Count Odo was the leader of the Parisian resistance during the Seige
	\item Ordinary townspeople helped repair the tower that was consistently damaged by the Northmen
	\item Emperor Charles fought them, but also used bargaining, giving them coinage and land.
\end{itemize}


\section*{Reading 8 : The Magyar Raids}

\subsection*{Overview}

\begin{itemize}
	\item The attacks of the Magyars/Hungarians were the last barbarian raids on Europe
	\item Passage written by Flodoard, a canon of Reims, France
\end{itemize}

\subsection*{Notes}

\begin{itemize}
	\item King Berengar led the Magyars to pillage the Italian city of Pavia in 1924
	\item Only 200 souls remained. 44 churches and 2 bishops were killed
	\item The invaders were paid off with 8 measures of silver
	\item The Magyars then crossed the Alps, were routed by Rudolf II so they entered Gothia. A plague in Gothia killed many
	\item In 933, the Magyars divided into three with one attacking Italy and another attacking Henry's Germany. With help of Bavaria, Saxony, Henry cut almost 36000 down
	\item In 955, King Otto went to fight the Magyars and won. Via alliances with the Sarmatians, Bohemians, Lotharingians, they were almost annihilated.
\end{itemize}

\subsection*{Question}

\textit{How far-ranging were the Hungarian incursions and what impact did they have on towns? Why did towns act as magnets for the invaders? Did towns or townspeople have anything to do with their defeat?}

\begin{itemize}
	\item The incursions reached into France, Germany and Italy.
	\item Towns act as magnets because they were not as well defened as cities while also containing vast resources to support the local population.
	\item TODO

\end{itemize}
\end{document}
