\documentclass[12pt]{article}

\usepackage{fullpage,url,xspace,enumerate,multirow,enumitem}
\usepackage[pdftitle={CLAS 205 Notes},%
pdfsubject={University of Waterloo, CLAS 205, Winter 2017},%
pdfauthor={Daniel Cardoza}]{hyperref}

\begin{document}

\begin{center}
	{\Large\bf CLAS 205 : Medieval Society}\\
	\vspace{2mm}
	{\Large\bf Winter 2017}\\
	\vspace{4mm}
	{Daniel Cardoza}\\
\end{center}

\section*{Lecture 1}
\setlength{\parindent}{0pt}

\textit{What sources do we have for learning about medieval society?}\\
We have manuscripts, buildings, statues/art/mosaics, surviving institutions, folklore and archaeological sites.\\

\textit{What is codicology}\\
The study of manuscript books on parchment as physical objects.\\

\textit{What is paleography?}\\
The study of ancient writing systems and decoding ancient manuscripts.\\

\textit{Why are sculptures, paintings and mosaics studied?}\\
Their quality and quantity describes the wellness of the period. Also, their content displays the culture of the time.\\

\textit{What is dendochronology?}\\
The study of dating tree rings.\\

\textit{What is numismatics?}\\
The study of coins and currency.\\

\textit{What is paleobotany?}\\
The study of ancient plants.\\

\textit{What is an aereal survey?}\\
A survey of a specific site done by air. Used by archaeological digs.\\

\textit{What is the nursery rhyme, 'Ring Around the Rosie' actually about?}\\
The Black Death in the Middle Ages.\\

\textit{What is the nursery rhyme London Bridge about?}\\
TODO\\

\textit{Name 5 medieval institutions that survive to this day}\\
Parliament, Charter of Rights, Universities, Monastic Houses, Catholic Church.\\

\textit{When was Rome at its greatest peak?}\\
 Under Emperor Trajan between 98-117CE.\\
 
\textit{Under what Emperor was Rome at its greatest peak?}\\
Emperor Trajan.\\

\textit{What was the population of Rome at its peak?}\\
Between 45-60 million people.\\

\textit{Name 6 things a complex society built in imperialism can have access to?}\\
Literacy, coinage, specialization of labour, redistribution of wealth, high quality goods AND economic mobility (moving between economic classes).\\

\textit{Name 6 internal and external threats the Roman Empire faced?}\\
Plague, incompetent rulers, usurpers, barbarian invasions, financial collapse and loss of civic pride.\\

\textit{What century marked the decline of Rome?}\\
The third century CE.

\textit{What characterized the Barracks Emperors?}\\
They were short lived, died violently and had seized power via their army.\\

\textit{How many claimants to the throne were there during the 3rd Century CE?}\\
There were 67.\\

\textit{Where did invasions come from during the 3rd Century CE?}\\
The North in Germani and from the East in Syria.\\

\textit{What is the word for migrations of German tribes?}\\
Valkerwundurung.\\

\textit{How does Rome solve manpower needs for the army?}\\
Germanic tribes brought into the empire are used more heavily.\\

\textit{What percentage of the population owned $\geq$ 90\% of the wealth?}\\
1500 families did.\\

\textit{What emperor created the Tetrarchy?}\\
Diocletian in 293 CE. This divided the Roman Empire into 2 parts.\\

\textit{What were the name of the 4 rulers of the tetrarchy?}\\
There were \textbf{2 Augusti} and \textbf{2 Caesars}.\\

\textit{How did the number of provinces changed during the tetrarchy?}\\
They went from 50 to 100+. Each province was divided into 12 dioceses, each controlled by a Vicar.\\

\textit{What were the two types of armed forces during Diocletian's military reforms?}\\
The Comitatenses and Limitanei. The Comitatenses are mobile, well trained where the Limitanei was infantry and not as well equipped.\\

\textit{What did Diocletian institude in 301CE?}\\
The "Wage and Price Controls" that dictated certain occupations as hereditary in order to control labour shortage problems.\\

\textit{What percentage of the Roman population were farmers at Diocletian's reign?}\\
About 85-90\%.\\

\textit{What was the name of the city members who had to pay taxes up front, and collect them from city members?}\\
Curialies.\\

\textit{When did Diocletian abdicate?}\\
305CE\\

\textit{When did Diocletian die?}\\
312CE\\

\textit{What happened after Diocletian died?}\\

The tetrarchy collapsed and there was civil war. Constantine I was victorious.\\

\textit{What was the name and date of the battle where Constantine used the symbols $\chi, \rho$?}\\
The battle of Milvian Ridge in 312 CE.\\

\textit{What does In Hoc Signo Vincas mean?}\\
In this sign, you will conquer\\

\textit{What did Constantine I enact in 313 CE to legalize Christianity?}\\
The Edict of Milan.\\

\textit{What did Constantine do in 321 CE regarding religious holidays?}\\
He made Sunday a public holiday.\\

\textit{What two heretical Christian groups existed at the time?}\\
Arianism and Donatism.\\

\textit{What city did Constantine I rename Constantinople in 324CE?}\\
Byzantium.\\

\textit{What council did Constantine I create to form religious Christian standards?}\\
The Council of Nicaea. \\

\textit{Was Constantine I baptized?}\\
Yes, on his deathbed in 327 CE.\\

\textit{Who was the last emperor to rule a united empire?}\\
Theodosius I, the Great.\\

\textit{How did Theodosius handle paganism?}\\
He outlawed it.\\

\textit{What barbarian group revolted in the Eastern Empire in 378-379 CE?}\\
The Goths, who killed the Eastern Emperor Valens at Adrianople.\\

\textit{What group from the East pushed all other barbarian groups Westward?}\\
The Huns.\\

\textit{What was an effect of the decline of the Empire?}\\
De-urbanization, loss of grain and agriculture production.\\

\textit{What was the last piece of antique Latin writing?}\\
Boethius's \textbf{Consolation of Philosophy}.\\

\textit{What was the name given to the ideal Roman Empire?}\\
Rome Aeterna.

\section*{Lecture 2}
\setlength{\parindent}{0pt}

\textit{Who were the first barbarians to settle in the Roman Empire?}
The Visigoths.\\

\textit{When was Britain evacuated by the Romans?}\\
400 CE.\\

\textit{Which two tribes sacked Rome in the 5th Century CE?}\\
$V^{2}$ aka Visigoths, Vandals\\

\textit{Where did the Merovingian Franks settle?}\\
In Gaul or modern France.\\

\textit{Where did the Vandals settle?}\\
In North Africa.\\

\textit{Where did the Ostrogoths settle?}\\
In Italy.\\

\textit{Where did the Visigoths settle?}\\
In Spain.\\

\textit{Where did the Angles, Saxons, and Jutes settle?}\\
In Great Britain.\\

\textit{When did Justinian reign and what is he known for?}\\
He reigned in 527-565CE. He is known for outlawing the Olympic games and pagan philosophy. He also created the Justinian law code. He married an actress Theodora.\\

\textit{Who was Justinian's best commander?}\\
It was Belisarius.\\

\textit{Who founded Islam and when?}\\
Mohammad of Mecca in 622\\

\textit{How far West did the Islamic Empire get?}\\
They made it to Tours, France.\\

\textit{What two groups did the Islamic faith split into?}\\
Into the Sunni and Shiites.\\

\textit{What two Islamic Empires were most powerful after the collapse?}\\
The Fatimids in Egypt and the Al-Andalus in Spain.\\

\textit{Who were the Merovingian Franks founded by?}\\
Merovich.\\

\textit{Who is the first recorded King of the Merovingian Franks?"}\\
It was Childeric.\\

\textit{Who succeeded Childeric?}\\
It was Clovis.\\

\textit{Who is the most important source of the period?}

It was Gregory of Tours with his 'The History of the Franks'.\\

\textit{The Merovingian Franks were the first barbarian kingdom to become <WHAT>?}\\
Catholic as other groups become Aryan and other Christian sects.\\

\textit{What dynasty took over from the Merovingians?}

The Carolingians.\\

\textit{Who founded the Carolingian Dynasty?}

It was Charles Martel.\\

\textit{Who was Charlemagne's biographer?}

It was Einhard.\\

\textit{Describe Charlemagne's conquest?}

\textit{Describe Charlemagne's stature?}

He was tall, had a belly, spoke many languages but was not literate. Lustful but pious.\\

\textit{Who are some of the scholars Charmlemagne imported?}

Einhard (his biographer), Alcuin of York, Paul the Deacon and Theodulf.

\textit{What were the two groupings of subjects in Carolingian education? What subjects belonged in each?}

The Trivium (grammer, rhetoric, logic) and Quadrivium (arithmetic, astronomy, geometry, music).

\textit{What two positions did each Carolingian district have?}

They each had a bishop and chief vassal called a Count.\\

\textit{What was the Missi Dominici?}
One was a noble and the other a clergyman to balance interest. They would ensure royal and religious decree were followed. They would also collect taxes for the royal treasury.Kept an eye on the local courts (vassals and bishops).
Means "those who were sent by the king". 

\textit{How often was a court visited by the Missi Dominici?}

At least 4x a year.\\

\textit{When was Charlemagne crowned Holy Roman Emperor and why?}

In 800CE for capturing Northern Italy from the Lombards. The Pope favoured Charlemagne over them.\\

\textit{Who were Charlamagne's three grandsons?}

They were Charles the Bald (got France), Lothar (Italy to North Sea) and Louis the German (HRE, Germany).\\

\textit{Why did some Northmen become vikings and leave their homeland?}

If a group lost in a civil war they had to leave. The North had become overpopulated and heirs with no fortune had to retrieve land, finances somehow. Their longboats enabled them to travel.\\

\textit{What were favoured viking targets?}

Monasteries because they were rich and not well defended.\\

\textit{When was Justinian's Law code rediscovered?}

It was rediscovered in 1050CE. \\

\section*{Lecture 3}

\textit{What two dynasties took over after the Carolingian's died out?}

The Capetian's in France and the ottonian's in Germany/HRE.\\

\textit{What is lay investiture?}

Alonging a non-church person to appoint bishops and Church personnel. The church considered this their power.\\

\textit{What is anti-simony?}

The opposite of simony which was buying church offices and roles/positions.\\

\textit{Describe the mutual dependency of church and state?}

States needed the moral authority of the Church but the Church needed protection via the states.\\

\textit{Describe the 3 various classes of crusade.}

The ones to the East, the Reconquista of Spain and the ones against the pagan tribes of Eastern Europe.\\

\textit{What were some overt motivations for crusades?}

Absolution from sin, liberating holy places and safeguarding pilgrimage routes.\\

\textit{Describe 3 hypotheses for the crusades?}

Landless nobles, rescue Jerusalem from heathens and economic imperialism under the guise of religious piety.

\textit{Who called the first crusade?}

Pope Urban II In Clermont, France. In 1095.\\

\textit{What group set out early on the first crusade? What happened?}

Peter the Hermit's group. All were slaughtered. (> 20000) \\

\textit{What was the most successful crusade?}

The first Crusade.\\

\textit{Name a difference between the first and second crusade?}

The second was led by Kings whereas the first by Dukes.\\

\textit{What prompted the second crusade?}

The seizing of Edessa by Muslim Forces in 1144. \\

\textit{When was the definitive loss of Jerusalem?}

In 1244. \\

\textit{What did the West get back from the crusades?}

They got books, paper and luxury goods from the East.\\

\textit{What 3 religious orders of Knights were created?}

The Templars, Hospitallers and Teutonic Knights.\\

\textit{Where did the Black Death originate and how did it spread to Europe?}

It originated in Central Asia and spread via trade routes.\\

\textit{Name 4 theories of what the Black Death was.}

\textit{How many people died in Europe from the Black Death?}

About 25-50 million.\\

\textit{What percentage of Europe's population died from the Black Death?}

Between 30-60\%.\\

\textit{When did the Black Death leave?}

In the 18th Century CE. \\

\textit{Where was a second Pope established?}

In Avignon, France.\\

\textit{Name 5 events that mark the event of the medieval ages?}

The Fall of Constantinople, Columbus, moveable type, expulsion of Muslims from Spain and the Reformation.\\

\textit{What word for monasticism stem from and what does it mean?}

It stems from monachos == lonely one.\\

\textit{Describe the origins of Monasticism in the east}

You had desert preachers in Syria, Anthony in Egypt.\\

\textit{Describe the origins of Monasticism in the West}

Martin of Tours was a bishop, exorcist and destroyer of country temples.\\

\textit{Who was Columbanus?}

He was an Irish monk with 12 companions setting up monasteries in France, Switzerland and Italy.\\

\textit{Describe the three types of monasteries.}

There were single-gender controled by an Abbot or Abbess. There were double monasteries where both sexes were under an Abbess. There were family monasteries with a patron family where descendants were the abbess or abbot.\\

\textit{Why did people want to go to monasteries?}
\begin{itemize}
	\item Be closed to God
	\item Outward display of contrition
	\item Escape from a tumultuous world
	\item Orphans
\end{itemize}

\textit{Were monks/nuns members of the laity or clergy?}

No. They were not ordained and were not laity as they lived ascetic lives.\\

\textit{Who founded the Benedictines?}

St. Benedict of Nursia with his sister St.Scholastica in the 530s.

\textit{Describe the sleeping situation at a Benedictine Monastery.}

They lived in rooms of 10-20 beds fully clothed. Youth and elders were mixed to prevent shenanigans.\\

\textit{Describe the meal situation at Benedictine monasteries.}

They were mainly vegetarian unless sick or old. Meals were done in silence except for reader.\\

\textit{What rations did people have at the monastery per day?}

A pint of wine, pound of bread. They would also have soup/porridge plus 1 fruit or vegetable.\\

\textit{What did the monks do other than pray?}

They performed manual labour, reading of sacred texts and more manual labour.\\

\textit{How many sets of clothing did monks have?}

They had 2 (so one could be worn while the other washed).\\

\textit{Describe the leadership of Benedictine monasteries.}

There was an abbot or abbes (chaste, sober, merciful). Sometimes, deans/provost were elected. Meritocracy was used.

\textit{Describe the 4 levels of rule enforcement}

\begin{enumerate}
	\item 2 Private verbal warnings
	\item 1 Public warning
	\item Excommunication
	\item Corporeal Punishment
	\item Explusion
\end{enumerate}

\textit{Were monks allowed to talk during meals?}

No, but guests could.\\

\textit{Were monks allowed to send letters?}

Not without explicit permission.\\

\textit{Were Benedictine monks allowed to have fun?}

Not really.\\

\textit{How were guests to be received at Benedictine Monastery?}

As 'Christ himself'. Especially pilgrims.\\

\textit{Was a monastery self-sustaining?}

Yes. They had mills, gardens, kitchens, baker and water to prevent travel. Each monastery had a doorkeeper versed in protocols.\\

\textit{What were some of the problems at Benedictine monasteries?}

Lack of discipline, influenced by secualr people and landowners interfered with the monasteries.\\

\textit{Who created the Benedictine Order at Cluny?}

William, Duke of Aquitaine and Abbot Berno.\\

\textit{Describe a Mendicant Order and how it differs from a Benedctine one.}

Rather than live in a monastery having no fun and not talking to Laity, they preached to communities and went abroad.
They lived a life of poverty and were presnt in all communities.\\

\textit{What does "Mendicant" mean?}

Begging.\\

\textit{Who founded the Franciscan Mendicant order?}

St. Francis of Assisi in 1209.\\

\section*{Lecture 4}

\textit{Who called for the 4th Crusade?}

Pope Innocent III.\\

\textit{What was silly about the 4th Crusade?}

Crusaders were excommunicated when attaching a Christian port but absolves them when it is taken. When they attack Constantinople, they were excommunicated again.
Then, the crusader king of Constantinople creates Latin states in Greece. Makes the patricha subservient to the Pope. Now they are absolved again.\\

\textit{What did crusaders return with from the 4th crusade?}

Lots of religious relics.\\

\textit{Who were the two leaders involved with the investiture controversy?}

Pope Gregory VII and King Henry IV of the Holy Roman Empire.\\

\textit{What did King Henry IV do after being excommunicated?}

He went barefoot to Pope's castle at Canossa.\\

\textit{Who founded the Mendicant Order of the Dominicans?}

St. Dominic in 1215. (around the same time as the Franciscans in 1209).\\

\textit{Describe the Dominican order}.

Church;s intellectual shock troops. Conversion via rational argument.\\

\textit{Why werre the Dominicans created?}

To deal with the Cathar Heresy.\\

\textit{What was the difference between Carthusians and Cistercians?}

The Carthusians were hermit communities while the Cistercians focussed on manual labour rather than prayer (unlike the Benedictines.)\\

\textit{Why were certain military orders established?}

To protect pilgrims, ensure access to holy places and propagate/defend the faith.\\

\textit{Name the 6 military-religious orders established}

The Templars, Hospitallers, Teutonic Knights, Order of Lazarus Knights of St. Thomas of Canterbury and the Knights of Montjoie.\\

\textit{Describe Hospitallers}

Their aim is to help poor and sick pilgrims.Roots of modern St. John's ambulance.\\

\textit{Describe the Templars.}

Founded in 1118 by Bernard of Clairvaux. Disbanded in 1307-1314. Posessions transferred to hospitallers.\\

\textit{Describe the Teutonic Knights}

German order founded in Palestine but moved to Hungary in 1211. Engaged in Baltic crusades against eastern-european tribes. Suffered defeat in 1410 and dissolved in 1525.\\

\textit{What is the name for the universal Christian commonwealth?}

Christendom.\\

\textit{What was the main principle of Christianity and Christendom?}

Eternal salvation > bodily prosperity.\\

\textit{When was the codifying of sacraments performed?}

4th Lateran Council by Pope Innocent III in 1215.\\

\textit{Describe the church hierarchy}

Pope -> Archbishops -> Bishops -> Priests.\\

\textit{Define a pilgrimage}

A trip to a Holy site.\\

\textit{What define a place as holy?}

A relic or a given event.\\

\textit{Did pilgrims start with Christianity?}

No, pagans did them as well.\\

\textit{Who went on pilgrimages?}

Everyone who could afford to.\\

\textit{What were the two types of pilgrim badges?}

Holy and secular.\\

\textit{What were the first badges sold to pilgrims?}

Scallop shells at Santiago di Compostela.\\

\textit{What is the name of a hearty, non-perishable bread eaten by pilgrims?}

Panforte.\\

\textit{What were 3 main pilgrimage sites?}

Jerusalem, Rome, Santiago di Compostela, Canterbury England.


\section*{Lecture 5}

\textit{What story gives us a good depiction of what a pilgrimage was like?}

Chaucer's Canterbury tales.\\

\textit{Why were relics in high deman?}

They healed people, people wee holier in their present AND they were mandatory for consecrating an altar in any Church from 787+ CE.\\

\textit{Briefly describe the two types of relics}

One type were ordinary objects that had come into contact with another person. The second were actual physical body parts of a saint.\\

\textit{When were type-A relics common?}

They were popular up until the 6th century CE.\\

\textit{When did type-B relics become popular?}

In the 8th Century. This was a direct result from the order in 787. Relics were needed more than ever and these were easiest to produce from the remains of Saints.

\textit{Relics related to what figures were the most common?}

Christ and Mary.\\

\textit{What event flooded the market with relics and caused the legitimacy of some to be questioned?}

The looting of Constantinople during the 4th Crusade.\\

\textit{What is a good indicator of the popularity of pilgrimages?}

The cemeteries constructed along pilgrimage routes with varied names.\\

\textit{What wee the 3 types of literature for early Germanic tribes?}

Prose, epic/heroic poetry and sermons.\\

\textit{What is the earliest dated vernacular literature?}

8th Century CE, with the exception of the Anglo-Saxons (6th-11th). Most of our records come from 12th Century CE.\\

\textit{What is the earliest extant-secular text from the Anglo-Saxons?}

Beowulf.\\

\textit{What is the name for the 'Sagas of the Norweigan Kings'?}

Heimskringla.\\

\textit{Who wrote the Heimskringla?}

Snorri Sturluson.\\

\textit{Who was the first European to develop printing by moveable type?}

Johannes Gutenburg.\\

\textit{What is papyrus made from?}

Overlapping layers of an Egyptian Reed plant. Used until 11th CE. \\

\textit{What was a role of papyrus called?}

A volumen.\\

\textit{What was parchment made from?}

Animal skin.\\

\textit{When was paper invented?}

2nd Century CE by the Chinese.\\

\textit{When was paper introduced to Europe?}

12th Century CE.\\



\section*{Reading 1 : The Grandeur of Rome}

\subsection*{Overview}

\subsection*{Notes}

\subsection*{Question}

\section*{Reading 7 : The Raids of the Northmen}

\subsection*{Overview}
\begin{itemize}
	\item In the 9th, 10th centuries, Arabs from the South, Northmen from Scandinavia and Magyars from the Eastern Steppe plagued the towns of Europe (England, France, Low Countries)
\end{itemize}

\subsection*{Notes}

\subsubsection*{The Early Raids of the Northmen, 834 -859}
\begin{itemize}
	\item Northmen pillaged Nantes in 834, killing the bishop and clergy
	\item Northmen pillaged along the Garonne until Toulouse
	\item With 100 ships, the Northmen assailed Paris after being unopposed. Charles the Bald gave them 7000lbs of (coin, resources) for them to leave. They did, but pillaged all the way to the coast
	\item Danish pirates pillaged Nantes in 853-854 pillaged the city of Nantes, chateau of Blois before being routed at Orleans. Routed by the bishop of Orleans and Chartres
	\item Tried to pillage Poitiers in 855 but were defeated by Aquitanians
	\item Pillaged Orleans in 856
	\item Pillaged along the Rhone in 859, settling on the island of Camargue. Ravaged up until Valence. Travelled to Italy afterward
\end{itemize}

\subsubsection*{The Siege of Paris, 885}
\begin{itemize}
	\item Sigfred, leader of the Danes, assailed Paris with 700 ships in 885
	\item Count Odo led the defence of Paris
	\item After the first day of fighting, the town's tower was damaged and repaired
	\item There was a plague inside the city. Count Odo went to seek Charles, the Frankish Emperor for help
	\item Count Odo came back with Emperor Charles the Fat. Charles allowed theNorthmean to have Sens and gave them 700lbs of silver.
\end{itemize}
	
\subsection*{Question}

\begin{center}
	\textit{Did the cities meet the Viking raids with any type of resistance? Who led the resistance in Paris? What role did ordinary townspeople play in the defense of Paris? How did Emperor Charles the Fat choose to deal with the attackers?}
\end{center}

\begin{itemize}
	\item Certain cities did, but the reading seems to suggest they pillaged most towns unopposed. 
	\item Count Odo was the leader of the Parisian resistance during the Seige
	\item Ordinary townspeople helped repair the tower that was consistently damaged by the Northmen
	\item Emperor Charles fought them, but also used bargaining, giving them coinage and land.
\end{itemize}


\section*{Reading 8 : The Magyar Raids}

\subsection*{Overview}

\begin{itemize}
	\item The attacks of the Magyars/Hungarians were the last barbarian raids on Europe
	\item Passage written by Flodoard, a canon of Reims, France
\end{itemize}

\subsection*{Notes}

\begin{itemize}
	\item King Berengar led the Magyars to pillage the Italian city of Pavia in 1924
	\item Only 200 souls remained. 44 churches and 2 bishops were killed
	\item The invaders were paid off with 8 measures of silver
	\item The Magyars then crossed the Alps, were routed by Rudolf II so they entered Gothia. A plague in Gothia killed many
	\item In 933, the Magyars divided into three with one attacking Italy and another attacking Henry's Germany. With help of Bavaria, Saxony, Henry cut almost 36000 down
	\item In 955, King Otto went to fight the Magyars and won. Via alliances with the Sarmatians, Bohemians, Lotharingians, they were almost annihilated.
\end{itemize}

\subsection*{Question}

\textit{How far-ranging were the Hungarian incursions and what impact did they have on towns? Why did towns act as magnets for the invaders? Did towns or townspeople have anything to do with their defeat?}

\begin{itemize}
	\item The incursions reached into France, Germany and Italy.
	\item Towns act as magnets because they were not as well defened as cities while also containing vast resources to support the local population.
	\item TODO

\end{itemize}


\section*{Reading 11 : The Origins of the Saxon Towns}

\subsection*{Overview}

\begin{itemize}
	\item Henry 1 was the first German King of the Saxon family/house
	\item After defeating the Magyars, Henry focused on taking territory from the Slavs to the East and fortifying new towns.
\end{itemize}

\subsection*{Notes}

\begin{itemize}
	\item After King Henry defeated the Magyars, he focused on defending his existing kingdom
	\item One out of 9 peasants moved into and helped buid fortifications. 
	\item One third of all produce was brought to these fortifications
	\item While these new fortified towns were being constructed, Henry attached the Slavs in Havel (Heveldi)
	\item He then took Dalmantia, Jahna and took Prague from the Bohemians
\end{itemize}

\subsection*{Question}

\textit{Which places did King Henry choose to make into cities and what steps did he take to accomplish this end? How do the origins of these Saxon towns compare to the town origins related in docs 9, 10?}

\begin{itemize}
	\item Allocating human resources in the form of $\frac{1}{9}$ peasants helped him fortify the new towns
	\item The allocation of one third of all crops to go to these towns made people flock to them
	\item TODO
\end{itemize}

\section*{Reading 13 - Granted of Privileges to the Castillians, Mozarabs and Franks of Toledo}

\subsection*{Overview}

\begin{itemize}
	\item Charter made in 1118 after King Alfonso II seized Toledo from the Moslems
	\item Describe 5 different groups residing in Toledo at the time : Castilians, Mozarabs, Franks, Galicians and Jews.
\end{itemize}

\subsection*{Notes}

\begin{itemize}
	\item Wants to re-establish a pact with the Franks, Mozarabs and Castillians still residing in the city
	\item All knights are exempted from tools at the gates on horses,mules
	\item Clergy, Christian keep hereditary goods which are exempt from tithes
	\item Any gifts given from the King should be divided evenly among the Knights of Toledo, Castilians, Galicians and Mozarabs
	\item If a Moor or Jew has a legal case against a Christian, they use a Christian jude
\end{itemize}

\subsection*{Question}

\textit{Who were the "knights" of Toledo and what privileges did they receive? What were they expected to do in exchange for these privileges? Did the different religious and ethnic groups in Toledo enjoy the same privileges and responsibilities? To what extent does this charter illustrate harmony or tension between the different ethnic and religious groups of Toledo?}

\begin{itemize}
	\item Knights were expected to go on Abnudba (guarding of animal herds on the frontier) or Fossatum (military conquest into muslim territory) in exchange for the benefits they received
	\item Not everyone had the same privileges or responsibilities. For example, the Franks and Jews were excluded from receiving benefits from the King while the other citizen groups did
\end{itemize}


\section*{Reading 15 - Archaeological Excavations in Tenth-Century York}

\subsection*{Overview}
\begin{itemize}
	\item York was initially a Roman Town but was captured by Vikings in 866.
	\item Viking maintained it until 954
	\item Archaeological finds at Coppergate in York shed light on York when the Northmen resided there
\end{itemize}

\subsection*{Notes}
\begin{itemize}
	\item One find was buildings with Wattle walls, a centreal hearth and benches
	\item Wattle paths
	\item Woodrunner/carpenter workshop
	\item Silver brooches. Also coppy or lead alloys. Most were lead
	\item Amber/Jet beads
	\item Forgery of Islamic coins
\end{itemize}

\subsection*{Question}
	\textit{None}
	
\section*{Reading 16 : The People of Cologne Rebel against their Archbishop}

\subsection*{Overview}

\subsection*{Notes}

\subsection*{Question}

\section*{Reading 17 : The Formation of a Commune at Laon, 1116}

\subsection*{Overview}
\begin{itemize}
	\item Describes the formation of a commune by townspeople fed up with their local ruler, the Bishop
	\item The first commune that was formed by clergy, nobles and townspeople was disbanded after 3 years
	\item The second commune that arose of mainly townspeople, grew tired of constant taxes/fees by the Bishop and revolted
\end{itemize}

\subsection*{Notes}

\subsection*{Question}

\textit{How does Guibert of Nogent define a "commune"? Did the interest groups involved in the formation of the commune fit with Guibert's definition? What role did oaths play in the formation of the commune and the revolt against Laon's lord, the bishop? What tactics did townspeople employ to secure their commune after the bishop revokes it? What did the bishop fear about the formation of the commune? How does this revolt compare with that in Cologne in terms of motivation, leadership, aims?}

\begin{itemize}
	\item A \textbf{commune} is a community where members only pay the annual customary head tax to their ruler, fees for breaking the law with no other financial burdens imposed.
	\item Essentially. It was the regular townspeople forming the commune, for the purpose of foregoing other financial fees throughout the year. The sum paid to form the commune went to the ruling/upper class of nobles and clergy.
	\item Establishing the commune involved the nobles to profess oaths, stating they would maintain their end of the agreement. The beginning of the revolt stems from the Bishop forcing the King and all parties involved in the creation of the commune, to break their oaths.
	\item After the bishop revoked the commune, the townspeople left their jobs and positions. They did not expect to have much wealth left after the ruling class was to tax them.
	\item TODO
	\item TODO
\end{itemize}


\section*{Reading 96 : Problems among the Clergy at Rouen Cathedral : 1248}

\subsection*{Overview}
\begin{itemize}
	\item Bishops regularly visited churches, cathedrals and monasteries to ensure proper religious conduct was followed.
	\item This reading describes a bishop's visit to a Cathedral in Rouen
\end{itemize}


\subsection*{Notes}

\subsection*{Question}

\textit{What impression does this inquiry give us of the activities of the cathedral clergy within the town of Rouen?}

\begin{itemize}
	\item The cathedral clergy in Rouen did not practice church conduct with proper rigour
	\item The clergy also engaged with the local townspeople often, mostly when performing activities forbidden by the church
	\item Certain cathedral members talked to women during choir, left choir without reason, said Psalms too fast etc.
	\item Many of the Masters and Sirs were incontinent , thieves, murderers, drunkedness, dicing and trading. 
\end{itemize}

\textit{What were considered the right relations between the clergy and townspeople?}
\begin{itemize}
	\item It appears that the interactions between the clergy and townspeople should be limited to the basics (food, water etc.) and items realted to the church. This can be seen because of the bishop's anger towards clergy trading, engaging with merchants and being drunkards.
\end{itemize}

\section*{Reading 97 : The Beguines of Ghent, 1328}

\subsection*{Overview}
\begin{itemize}
	\item Beguine is a name for women devoting themselves to Gods but who were not nuns. Unlike Nuns, they were able to go about normal lives.
	\item Beguines lived in Beguinages
	\item This reading is from an inquiry into the Beguinage in Ghent.
\end{itemize}

\subsection*{Notes}

\subsection*{Question}

\textit{Is there any evidence of why so many urban women were drawn to the Beguine movement?}
\begin{itemize}
	\item There were many women who did not have husbands and could not enter a monastery because of capacity. 
	\item Women were drawn to beguinages because they could remain chase but have food, shelter and community
	\item Were there more women at this time because of a war and the men were gone?
\end{itemize}

\textit{To what extent did Beguines participate in town life?}
\begin{itemize}
	\item The principal mistress of the Beguinage ensured beguines did not stay in town for a night without permission and could not leave for an hour.
	\item They are not intended to meet suspect people when they do go into town
	\item The passage paints a lack of involvement in town life
\end{itemize}

\textit{Is there any evidence of how the townspeople might have viewed these religious women who lived in the world, rather than a convent?}
\begin{itemize}
	\item TODO
\end{itemize}

\section*{Reading 98 - The Good Canon of Cologne}

\subsection*{Overview}
\begin{itemize}
	\item Describes the virtue of a Canon in order to guide and train novices becoming monks
\end{itemize}

\subsection*{Question}
\textit{What types of social, religious and charitable interactions did the canon have with the townspeople of Cologne?}
\begin{itemize}
	\item He consistently gave away food, coin and others to the impoverished
	\item He brought the people to feasts as guests
	\item TODO
\end{itemize}

\textit{What does the passage tell us about attitudes towards the poor?}
\begin{itemize}
	\item It tells us that most were not really cared for by the Church, making what the Canon did exceptional.
	\item The meeting with Abbess confirms that certain clergy were opulent and not pious like the canon
\end{itemize}

\textit{Why does the canon criticize the abbess and how does his concern compare to the complaints of the reforming bishop of Rouen in doc. 96?}
\begin{itemize}
	\item He criticizes her because she and her group are dressed nicely whereas he is surrounded by the poor. The money she used on clothing and luxury, could have been given to the poor
	\item The complaints are similar but also different. The Bishop of Rouen complained about the rigour of church practice and how certain members traded, were incontinent, and drunkards. However, he never commented on their morality, like the Canon does to the Abbess. The canon's complaints are moral complaints and align themself with the teachings of the Church. The Bishop's complaints were about Church policy and conduct, but not the specific teachings in the Bible.
\end{itemize}

\section*{Reading 99 : A Popular Franciscan Preacher in Paris : 1429}

\section*{Overview}

\begin{itemize}
	\item Describes the charismatic nature of a Franciscan preacher during the Hundred Years
\end{itemize}

\subsection*{Notes}

\subsection*{Question}

\textit{What does this passage tell us about sermonizing and its impact in medieval towns?}
\begin{itemize}
	\item It tells us that it was able to have a large impact when the preacher was charismatic and consistently preached. Brother Richard was very frequent with his preaching and as a result, was able to talk to a large audience.
\end{itemize}
\textit{Why were people compelled to attend these long sermons?}
\begin{itemize}
	\item People may have felt compelled to attend because he travelled to Paris specifcally for the purpose of preaching. He may have been known and respected as well, making people more inclined to hear him talk.
	\item As well, the writer wrote this during the tumultuous years of the Hundred Year's War. Perhaps this caused people to want to be more pious in order to end the suffering the conflict caused?
\end{itemize}
\textit{How did those most influenced by the sermons react and why did they take the actions they did?}
\begin{itemize}
	\item People destroyed games deemed covetous like dice, cards, balls and sick that could cause anger or swearing.
	\item Women removed fine clothing, head-geat, stuffing and other items for pure purpose of vanity. This affected women of all social classes
	\item People were sad and cried when he announced he had to leave.
	\item Over 6000 people wished to attend his final sermon, although he did not arrive to give it.
\end{itemize}

\section*{Reading 102 : The Religious Fraternity of St.Katherine at Norwich, 1389}

\begin{itemize}
	\item Religious guilds/groups like that dedicated to St.Katherine in Norwich provided benefits to members
	\item In exchange for money, candles were provided for prayer and sermons for recently deceased humans were provided
\end{itemize}

\subsection*{Notes}

\subsection*{Question}
\textit{Why would people join this guild?}
\begin{itemize}
	\item For the services it offers, especially in times of misfortune. This includes the death of a family member, being struck with poverty.
\end{itemize}

\textit{What services or benefits did it have to offer?}
\begin{itemize}
	\item They offered mass for a deceased family member
	\item They offered to carry the dead, if permitted, to Norwich for a ceremony. If not, they bured the dead in place and still celebrated mass.
	\item They offered welfare if a family was struck with poverty
	\item They offered community. Members ate together on their guild day, wore the same clothing and were together often as a result of the above offered services
\end{itemize}

\textit{How did the guild help ensure a sense of solidarity among its members}
\begin{itemize}
	\item Solidarity was created as a result of the common clothing, consistent gatherings and the services provided.
	\item Being a member of the guild helped out a person in their time of need, giving a sense of community.
\end{itemize}

\section*{Reading 105 : A Maison-Dieu in Pontoise}

\subsection*{Overview}

Maison-Dieu means house of God in French. These establishments took in the poor, ill and pregnant who had nowhere else to go. This reading describes the statues and regulations of how the hospital should be run.

\subsection*{Notes}

\subsection*{Questions}

\textit{What types of patients entered this "hospital"?}

People who were will with both surface and serious maladies. Pregnant women were also said to come to the House.\\

\textit{What rules did the statutes impose on the patients and upon the hospital administration?}

Very strict rules regarding haircuts, types of skills, going into town to eat or drink and more.\\

\textit{Why is there such an emphasis on the religious and moral behaviour of the brothers and sisters of the hospital order, as well as of the patients?}

The existence of the establishment is owed to the Church, so it must rigorously uphold the teachings of the Church. Also, the importance of baptism for newborns and confessions for those about to die were important also. One could argue that if the discipline of the people servicing the house wained, they may received less funding and financial support. \\

\section*{Reading 106 : The Distribution of Alms in Florence}

\subsection*{Overview}

This page describes the Orsanmichele which was the richest charity in Florence. It gives insight into how charitable resources were distributed. Note an unskileld labour were to make 7-10 shillings a day.

\subsection*{Notes}

\subsection*{Question}

\textit{Who -- in terms of gender, marital status, and age -- was most likely to receive help from this private charity?}

To women who are pregnant or caring for a family. Those who were married or widowed were treated well. The younger you were, the more you received as well. Older women were also treated very well it appears, especially if they cannot take care of themself.

\textit{What were the circumstances of most recipients of charity?}

Most were to help support a family for a lone wife with many children or who was pregnant.  Most people given charity could not support themselves. This included the old/infirm as well as the young.


\section*{Reading 107 : Charitable Bequests in Siena}

\subsection*{Overview}

\subsection*{Notes}

\subsection*{Question}

\textit{What were the types of religious orders, institutions, and individuals who were included in the bequests of these four Siennese citizens?}

\textit{What differences and similarities do you see between the choices made by the testators in deciding who the beneficiaries of their charity would be and how much they received?}

\textit{What do the testators' choices of burial tell you about their religious affiliations?}

\textit{Which religious groups tended to receive larger or smaller amounts of money?}

\textit{What do the nature of these legacies -- some in cash, and some not -- indicate about medieval urban religiosity?}


\end{document}