\documentclass[12pt]{article}

\usepackage{fullpage,url,xspace,enumerate,multirow,enumitem}
\usepackage[pdftitle={CLAS 205 Notes},%
pdfsubject={University of Waterloo, CLAS 205, Winter 2017},%
pdfauthor={Daniel Cardoza}]{hyperref}

\begin{document}

\begin{center}
	{\Large\bf CLAS 205 : Medieval Society}\\
	\vspace{2mm}
	{\Large\bf Winter 2017}\\
	\vspace{4mm}
	{Daniel Cardoza}\\
\end{center}

\section*{Reading 1 : The Grandeur of Rome}

\subsection*{Overview}

\subsection*{Notes}

\subsection*{Question}

\section*{Reading 7 : The Raids of the Northmen}

\subsection*{Overview}
\begin{itemize}
	\item In the 9th, 10th centuries, Arabs from the South, Northmen from Scandinavia and Magyars from the Eastern Steppe plagued the towns of Europe (England, France, Low Countries)
\end{itemize}

\subsection*{Notes}

\subsubsection*{The Early Raids of the Northmen, 834 -859}
\begin{itemize}
	\item Northmen pillaged Nantes in 834, killing the bishop and clergy
	\item Northmen pillaged along the Garonne until Toulouse
	\item With 100 ships, the Northmen assailed Paris after being unopposed. Charles the Bald gave them 7000lbs of (coin, resources) for them to leave. They did, but pillaged all the way to the coast
	\item Danish pirates pillaged Nantes in 853-854 pillaged the city of Nantes, chateau of Blois before being routed at Orleans. Routed by the bishop of Orleans and Chartres
	\item Tried to pillage Poitiers in 855 but were defeated by Aquitanians
	\item Pillaged Orleans in 856
	\item Pillaged along the Rhone in 859, settling on the island of Camargue. Ravaged up until Valence. Travelled to Italy afterward
\end{itemize}

\subsubsection*{The Siege of Paris, 885}
\begin{itemize}
	\item Sigfred, leader of the Danes, assailed Paris with 700 ships in 885
	\item Count Odo led the defence of Paris
	\item After the first day of fighting, the town's tower was damaged and repaired
	\item There was a plague inside the city. Count Odo went to seek Charles, the Frankish Emperor for help
	\item Count Odo came back with Emperor Charles the Fat. Charles allowed theNorthmean to have Sens and gave them 700lbs of silver.
\end{itemize}
	
\subsection*{Question}

\begin{center}
	\textit{Did the cities meet the Viking raids with any type of resistance? Who led the resistance in Paris? What role did ordinary townspeople play in the defense of Paris? How did Emperor Charles the Fat choose to deal with the attackers?}
\end{center}

\begin{itemize}
	\item Certain cities did, but the reading seems to suggest they pillaged most towns unopposed. 
	\item Count Odo was the leader of the Parisian resistance during the Seige
	\item Ordinary townspeople helped repair the tower that was consistently damaged by the Northmen
	\item Emperor Charles fought them, but also used bargaining, giving them coinage and land.
\end{itemize}


\section*{Reading 8 : The Magyar Raids}

\subsection*{Overview}

\begin{itemize}
	\item The attacks of the Magyars/Hungarians were the last barbarian raids on Europe
	\item Passage written by Flodoard, a canon of Reims, France
\end{itemize}

\subsection*{Notes}

\begin{itemize}
	\item King Berengar led the Magyars to pillage the Italian city of Pavia in 1924
	\item Only 200 souls remained. 44 churches and 2 bishops were killed
	\item The invaders were paid off with 8 measures of silver
	\item The Magyars then crossed the Alps, were routed by Rudolf II so they entered Gothia. A plague in Gothia killed many
	\item In 933, the Magyars divided into three with one attacking Italy and another attacking Henry's Germany. With help of Bavaria, Saxony, Henry cut almost 36000 down
	\item In 955, King Otto went to fight the Magyars and won. Via alliances with the Sarmatians, Bohemians, Lotharingians, they were almost annihilated.
\end{itemize}

\subsection*{Question}

\textit{How far-ranging were the Hungarian incursions and what impact did they have on towns? Why did towns act as magnets for the invaders? Did towns or townspeople have anything to do with their defeat?}

\begin{itemize}
	\item The incursions reached into France, Germany and Italy.
	\item Towns act as magnets because they were not as well defened as cities while also containing vast resources to support the local population.
	\item TODO

\end{itemize}


\section*{Reading 11 : The Origins of the Saxon Towns}

\subsection*{Overview}

\begin{itemize}
	\item Henry 1 was the first German King of the Saxon family/house
	\item After defeating the Magyars, Henry focused on taking territory from the Slavs to the East and fortifying new towns.
\end{itemize}

\subsection*{Notes}

\begin{itemize}
	\item After King Henry defeated the Magyars, he focused on defending his existing kingdom
	\item One out of 9 peasants moved into and helped buid fortifications. 
	\item One third of all produce was brought to these fortifications
	\item While these new fortified towns were being constructed, Henry attached the Slavs in Havel (Heveldi)
	\item He then took Dalmantia, Jahna and took Prague from the Bohemians
\end{itemize}

\subsection*{Question}

\textit{Which places did King Henry choose to make into cities and what steps did he take to accomplish this end? How do the origins of these Saxon towns compare to the town origins related in docs 9, 10?}

\begin{itemize}
	\item Allocating human resources in the form of $\frac{1}{9}$ peasants helped him fortify the new towns
	\item The allocation of one third of all crops to go to these towns made people flock to them
	\item TODO
\end{itemize}

\section*{Reading 13 - Granted of Privileges to the Castillians, Mozarabs and Franks of Toledo}

\subsection*{Overview}

\begin{itemize}
	\item Charter made in 1118 after King Alfonso II seized Toledo from the Moslems
	\item Describe 5 different groups residing in Toledo at the time : Castilians, Mozarabs, Franks, Galicians and Jews.
\end{itemize}

\subsection*{Notes}

\begin{itemize}
	\item Wants to re-establish a pact with the Franks, Mozarabs and Castillians still residing in the city
	\item All knights are exempted from tools at the gates on horses,mules
	\item Clergy, Christian keep hereditary goods which are exempt from tithes
	\item Any gifts given from the King should be divided evenly among the Knights of Toledo, Castilians, Galicians and Mozarabs
	\item If a Moor or Jew has a legal case against a Christian, they use a Christian jude
\end{itemize}

\subsection*{Question}

\textit{Who were the "knights" of Toledo and what privileges did they receive? What were they expected to do in exchange for these privileges? Did the different religious and ethnic groups in Toledo enjoy the same privileges and responsibilities? To what extent does this charter illustrate harmony or tension between the different ethnic and religious groups of Toledo?}

\begin{itemize}
	\item Knights were expected to go on Abnudba (guarding of animal herds on the frontier) or Fossatum (military conquest into muslim territory) in exchange for the benefits they received
	\item Not everyone had the same privileges or responsibilities. For example, the Franks and Jews were excluded from receiving benefits from the King while the other citizen groups did
\end{itemize}


\section*{Reading 15 - Archaeological Excavations in Tenth-Century York}

\subsection*{Overview}
\begin{itemize}
	\item York was initially a Roman Town but was captured by Vikings in 866.
	\item Viking maintained it until 954
	\item Archaeological finds at Coppergate in York shed light on York when the Northmen resided there
\end{itemize}

\subsection*{Notes}
\begin{itemize}
	\item One find was buildings with Wattle walls, a centreal hearth and benches
	\item Wattle paths
	\item Woodrunner/carpenter workshop
	\item Silver brooches. Also coppy or lead alloys. Most were lead
	\item Amber/Jet beads
	\item Forgery of Islamic coins
\end{itemize}

\subsection*{Question}
	\textit{None}
	
\section*{Reading 16 : The People of Cologne Rebel against their Archbishop}

\subsection*{Overview}

\subsection*{Notes}

\subsection*{Question}

\section*{Reading 17 : The Formation of a Commune at Laon, 1116}

\subsection*{Overview}
\begin{itemize}
	\item Describes the formation of a commune by townspeople fed up with their local ruler, the Bishop
	\item The first commune that was formed by clergy, nobles and townspeople was disbanded after 3 years
	\item The second commune that arose of mainly townspeople, grew tired of constant taxes/fees by the Bishop and revolted
\end{itemize}

\subsection*{Notes}

\subsection*{Question}

\textit{How does Guibert of Nogent define a "commune"? Did the interest groups involved in the formation of the commune fit with Guibert's definition? What role did oaths play in the formation of the commune and the revolt against Laon's lord, the bishop? What tactics did townspeople employ to secure their commune after the bishop revokes it? What did the bishop fear about the formation of the commune? How does this revolt compare with that in Cologne in terms of motivation, leadership, aims?}

\begin{itemize}
	\item A \textbf{commune} is a community where members only pay the annual customary head tax to their ruler, fees for breaking the law with no other financial burdens imposed.
	\item Essentially. It was the regular townspeople forming the commune, for the purpose of foregoing other financial fees throughout the year. The sum paid to form the commune went to the ruling/upper class of nobles and clergy.
	\item Establishing the commune involved the nobles to profess oaths, stating they would maintain their end of the agreement. The beginning of the revolt stems from the Bishop forcing the King and all parties involved in the creation of the commune, to break their oaths.
	\item After the bishop revoked the commune, the townspeople left their jobs and positions. They did not expect to have much wealth left after the ruling class was to tax them.
	\item TODO
	\item TODO
\end{itemize}


\section*{Reading 96 : Problems among the Clergy at Rouen Cathedral : 1248}

\subsection*{Overview}
\begin{itemize}
	\item Bishops regularly visited churches, cathedrals and monasteries to ensure proper religious conduct was followed.
	\item This reading describes a bishop's visit to a Cathedral in Rouen
\end{itemize}


\subsection*{Notes}

\subsection*{Question}

\textit{What impression does this inquiry give us of the activities of the cathedral clergy within the town of Rouen?}

\begin{itemize}
	\item The cathedral clergy in Rouen did not practice church conduct with proper rigour
	\item The clergy also engaged with the local townspeople often, mostly when performing activities forbidden by the church
	\item Certain cathedral members talked to women during choir, left choir without reason, said Psalms too fast etc.
	\item Many of the Masters and Sirs were incontinent , thieves, murderers, drunkedness, dicing and trading. 
\end{itemize}

\textit{What were considered the right relations between the clergy and townspeople?}
\begin{itemize}
	\item It appears that the interactions between the clergy and townspeople should be limited to the basics (food, water etc.) and items realted to the church. This can be seen because of the bishop's anger towards clergy trading, engaging with merchants and being drunkards.
\end{itemize}

\section*{Reading 97 : The Beguines of Ghent, 1328}

\subsection*{Overview}
\begin{itemize}
	\item Beguine is a name for women devoting themselves to Gods but who were not nuns. Unlike Nuns, they were able to go about normal lives.
	\item Beguines lived in Beguinages
	\item This reading is from an inquiry into the Beguinage in Ghent.
\end{itemize}

\subsection*{Notes}

\subsection*{Question}

\textit{Is there any evidence of why so many urban women were drawn to the Beguine movement?}
\begin{itemize}
	\item There were many women who did not have husbands and could not enter a monastery because of capacity. 
	\item Women were drawn to beguinages because they could remain chase but have food, shelter and community
	\item Were there more women at this time because of a war and the men were gone?
\end{itemize}

\textit{To what extent did Beguines participate in town life?}
\begin{itemize}
	\item The principal mistress of the Beguinage ensured beguines did not stay in town for a night without permission and could not leave for an hour.
	\item They are not intended to meet suspect people when they do go into town
	\item The passage paints a lack of involvement in town life
\end{itemize}

\textit{Is there any evidence of how the townspeople might have viewed these religious women who lived in the world, rather than a convent?}
\begin{itemize}
	\item TODO
\end{itemize}

\section*{Reading 98 - The Good Canon of Cologne}

\subsection*{Overview}
\begin{itemize}
	\item Describes the virtue of a Canon in order to guide and train novices becoming monks
\end{itemize}

\subsection*{Question}
\textit{What types of social, religious and charitable interactions did the canon have with the townspeople of Cologne?}
\begin{itemize}
	\item He consistently gave away food, coin and others to the impoverished
	\item He brought the people to feasts as guests
	\item TODO
\end{itemize}

\textit{What does the passage tell us about attitudes towards the poor?}
\begin{itemize}
	\item It tells us that most were not really cared for by the Church, making what the Canon did exceptional.
	\item The meeting with Abbess confirms that certain clergy were opulent and not pious like the canon
\end{itemize}

\textit{Why does the canon criticize the abbess and how does his concern compare to the complaints of the reforming bishop of Rouen in doc. 96?}
\begin{itemize}
	\item He criticizes her because she and her group are dressed nicely whereas he is surrounded by the poor. The money she used on clothing and luxury, could have been given to the poor
	\item The complaints are similar but also different. The Bishop of Rouen complained about the rigour of church practice and how certain members traded, were incontinent, and drunkards. However, he never commented on their morality, like the Canon does to the Abbess. The canon's complaints are moral complaints and align themself with the teachings of the Church. The Bishop's complaints were about Church policy and conduct, but not the specific teachings in the Bible.
\end{itemize}

\section*{Reading 99 : A Popular Franciscan Preacher in Paris : 1429}

\section*{Overview}

\begin{itemize}
	\item Describes the charismatic nature of a Franciscan preacher during the Hundred Years
\end{itemize}

\subsection*{Notes}

\subsection*{Question}

\textit{What does this passage tell us about sermonizing and its impact in medieval towns?}
\begin{itemize}
	\item It tells us that it was able to have a large impact when the preacher was charismatic and consistently preached. Brother Richard was very frequent with his preaching and as a result, was able to talk to a large audience.
\end{itemize}
\textit{Why were people compelled to attend these long sermons?}
\begin{itemize}
	\item People may have felt compelled to attend because he travelled to Paris specifcally for the purpose of preaching. He may have been known and respected as well, making people more inclined to hear him talk.
	\item As well, the writer wrote this during the tumultuous years of the Hundred Year's War. Perhaps this caused people to want to be more pious in order to end the suffering the conflict caused?
\end{itemize}
\textit{How did those most influenced by the sermons react and why did they take the actions they did?}
\begin{itemize}
	\item People destroyed games deemed covetous like dice, cards, balls and sick that could cause anger or swearing.
	\item Women removed fine clothing, head-geat, stuffing and other items for pure purpose of vanity. This affected women of all social classes
	\item People were sad and cried when he announced he had to leave.
	\item Over 6000 people wished to attend his final sermon, although he did not arrive to give it.
\end{itemize}

\section*{Reading 102 : The Religious Fraternity of St.Katherine at Norwich, 1389}

\begin{itemize}
	\item Religious guilds/groups like that dedicated to St.Katherine in Norwich provided benefits to members
	\item In exchange for money, candles were provided for prayer and sermons for recently deceased humans were provided
\end{itemize}

\subsection*{Notes}

\subsection*{Question}
\textit{Why would people join this guild?}
\begin{itemize}
	\item For the services it offers, especially in times of misfortune. This includes the death of a family member, being struck with poverty.
\end{itemize}

\textit{What services or benefits did it have to offer?}
\begin{itemize}
	\item They offered mass for a deceased family member
	\item They offered to carry the dead, if permitted, to Norwich for a ceremony. If not, they bured the dead in place and still celebrated mass.
	\item They offered welfare if a family was struck with poverty
	\item They offered community. Members ate together on their guild day, wore the same clothing and were together often as a result of the above offered services
\end{itemize}

\textit{How did the guild help ensure a sense of solidarity among its members}
\begin{itemize}
	\item Solidarity was created as a result of the common clothing, consistent gatherings and the services provided.
	\item Being a member of the guild helped out a person in their time of need, giving a sense of community.
\end{itemize}


\end{document}